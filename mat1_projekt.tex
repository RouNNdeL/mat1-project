\documentclass[a4paper, 12pt]{article}

\usepackage{float}
\usepackage[T1]{fontenc}
\usepackage{lmodern}
\usepackage[polish]{babel}
\usepackage[utf8]{inputenc}
\usepackage{polski}
\usepackage{hyperref}
\usepackage{amsmath}
\usepackage{mathtools}
\usepackage{ulem}
\usepackage{bm}
\usepackage{graphicx}
\usepackage{listings}

\usepackage{amsmath}
\usepackage{amssymb}
\usepackage{amsfonts}

\begin{document}

\title{ Faktoryzacja liczb całkowitych - część teoretyczna}
    \author{Krzysztof Zdulski, Kazimierz Kochan, Monika Lewandowska}
    \date{\today}
    \maketitle
    
\tableofcontents
\newpage

\section{Dowód 1 - Twierdzenie o dzieleniu z resztą}

Pokazać, że dla dowolnych liczb całkowitych a i b różnych od 0 istnieją jednoznacznie wyznaczone liczby całkowite q i r takie, że  

\(a=q*b+r,\ 0 \leq r<|b|\)

\subsection {Dowód istnienia liczb q i r}
\(a=q*b+r\) , \(0 \leq r<|b|\)

q- iloraz całkowity

r- reszta z dzielenia
\bigbreak
Załóżmy, że \(b>0\)
\begin{itemize}
	\item \(q=[\frac{a}{b}]\) \(\wedge \) \(r=a-b*q\)
	\item \(q\ \leq \frac{a}{b} < q+1\) | \(* b\)
	\item \(b*q\ \leq a \) < \(b*q +b \)
	\item Z tego wynika, że \(0 \leq [ r=a - b*q\)] < \( b=|b|\)
\end{itemize}
\bigbreak
Gdy \(b<0\)

\begin{itemize}
	\item \(q= - [\frac{a}{|b|}]\) \(\wedge \) \(r=a-b*q\)
\end{itemize}
i dalej analogicznie jak w przypadku \(b>0\) 

\subsection{Dowód, jednoznaczności wyboru pary liczb q i r}
\begin{itemize}

\item Załóżmy, że istnieją liczby całkowite \(q_{1}\), \(q_{2}\), \(r_{1}\) i \(r_{2}\) takie, że

 \(0 \leq\ r_{1}, r_{2}<|b|\) \(\wedge \) \(a= q_{1}*b + r_{1} = q_{2}*b + r_{2}\)
\item \(r_{2}-r_{1} = b*(q_{2}-q_{1})\)
\item Gdyby \(r_{2} - r_{1} \neq 0\) to \( |b|<|r_{2} - r_{1}|\leq max\{r_{1},r_{2}\}< |b|\)
,co prowadzi do sprzeczności

\item Więc \(r_{2}-r_{1}=0\), czyli \(r_{1}=r_{2}\) 

\item Stąd \((q_{2}-q_{1})*b=0\), zatem \((q_{2}-q_{1})=0\), więc \(q_{1}=q_{2}\), bo \(b \neq 0\)
\end{itemize}

Co potwierdza jednoznaczność wyboru pary liczb q i r.

\section{Dowód 2 - \(NWD(a,b)=ax+by\)} 

Pokazać, że istnieją liczby całkowite x i y takie, że \(NWD(a,b)=a*x+b*y\), gdzie $a,b \in \mathbb{Z}$ , z których conajmniej jedna jest różna od 0 
\begin{itemize}

\item Gdy \(a=0=b\) to \(0=a*0+b*0\) jest NWD(a,b), załóżmy więc, że a lub b jest różne od 0

\item Niech \( I=\{ \) \(k*a+l*b\) \( \wedge \) k,l \(\in \) \(\mathbb{Z} \) \(\cap\ \) \(N_{+}\} \)

\item Ponieważ \(|a|= sign(a)*a+0*b\) \( \wedge\ \) \(|b|=0*b+sign(b)*b\) 

Zatem I nie jest zbiorem pustym

\item Niech \( d=min\{I\} \) \( \wedge \) k,l \( \in \) \(\mathbb{Z}\ \) \( \wedge\ \) \(d=k*a+l*b\)

\item Należy pokazać, że \(d=NWD(a,b)\); wiadomo, że \(d \geq 0\)

\item Dla c \(\in \mathbb{Z}\ \) , \ \(c|a \wedge\ c|b \implies c|d\)

\item Należy udowodnić, że \(d|a \wedge\ d|b\)

\item Liczbę a można zapisać w postaci: \(a= q_{ad}*d\ +   r_{ad}\) (dowód nr. 1)

\item Zatem \(r_{ad} = a - q_{ad} *d = a- q_{ad}*(k*a+l*b) = a-q_{ad}*k*a- q_{ad}*l*b = 
a*(1-k*q_{ad})- q_{ad}*l*b\)
\item \(r_{ad}<d\), więc \(r_{ad} \notin I \), bo \(d=min \{I\}\) 
\item \(r_{ad} \geq 0\), więc z def. zbioru I można wnioskować, że \(r_{ad} = 0\)
\item Zatem d|a i analogicznie d|b, więc d=NWD(a,b) 
\end{itemize}
\newpage
\section{Dowód 3 - \(NWD(a,b)=1\) ,  \(a|c \)  \(\wedge\) \(b|c\) \(\implies\) \(ab|c\)} 
Pokazać, że jeżeli liczby a i b są względnie pierwsze, to \(a|c \wedge b|c \implies ab|c\), gdzie \( a,b,c \in \mathbb{Z} \)
\begin{itemize}

\item Z definicji zbioru liczb względnie pierwszych wynika, że NWD(a,b)=1, gdy a i b są różne

\item Skoro NWD(a,b)=1 wiemy, że (Dowód 2) istnieje taki x,y, że

 \(a*x+b*y=1\)

\item Zatem \(a*c*x+b*c*y=c\)

\item Z założenia b|c, więc ab|ac

\item Analogicznie, gdy a|c to ab|bc

\item Skoro ab dzieli całkowicie zarówno ac jak i bc to dzieli również wyrażenie \(a*c*x+b*c*y=c\)

\item Rozpatrując przypadek dla dowolnej liczby całkowitej m takiej, że \(c=m*a\) wiemy, że b|ma oraz NWD(b,a)=1 to prowadzi do stwierdzenia, że b|m

\item Dla dowolnej liczby całkowitej n takiej, że \(m=b*n\)   

\( c=a*m=a*b*n \), a zatem \(ab|c\)
\end{itemize}
\newpage
\section{Dowód 4 - \( p|ab\implies p|a \vee p|b \)}
Pokazać, że gdy a,b \( \in \) \(\mathbb{Z}\)
\( \wedge \) p jest liczbą pierwszą \( p|ab \) \implies \( p|a \vee\ p|b \)
\bigbreak
\begin{itemize}
	\item Załóżmy, że \( p|ab \) , ale \(p \nmid a\) \(\wedge\ \) \(p \nmid b\)
	\item Więc \(NWD(p,a)=1\) i \(NWD(p,b)=1 \) (są to pary liczb względnie pierwszych)
	\item Można zatem zapisać, że 
	
	\( x_{1}*p+ y_{1}*a=1 \)
	
	 \( x_{2}*p+y_{2}*b=1 \)
	 
	 Co wynika z dowodu nr 2
	 \item Mnożymy obustronnie równania i otrzymujemy:
	 
	 \(x_{1}*p*x_{2}*p+x_{1}*p*y_{2}*b+y_{1}*a*x_{2}*p+y_{1}*a*y_{2}*b=1 \)
	 
	 po przekształceniu
	
	 \( p*(x_{1}*x_{2}*p+x_{1}*y_{2}*b+y_{1}*a*x_{2})+ a*b(y_{1}*y_{2})=1\)
	 \item Z tego wynika, że \(NWD(p,ab) \)=1, co prowadzi do sprzeczności z założeniem, że p|ab
	 \item Zatem, aby założenie p|ab było prawdziwe to p|a \(\vee \) p|b 
\end{itemize}
\newpage
\section{Dowód 5 - Każda liczba naturalna n daje się przedstawić jako skończony iloczyn samych liczb pierwszych dla \(n \geq 2\) }
Pokazać, że każdą liczbę całkowitą \(n \geq 2\) można jednoznacznie przedstawić w postaci \(n= p_{1}^{a1}*p_{2}^{a2}*...p_{k}^{a_{k}}\), gdzie \(p_{1},p_{2},...,p_{k}\) są liczbami pierwszymi, \(p_{1}<p_{2}<...<p_{k}\) oraz \(a_{1},a_{2}...,a_{k} \in \mathbb{N}\) 
\bigbreak
Inaczej: Każda liczba naturalna daje się przedstawić jako skończony iloczyn samych liczb pierwszych.
\bigbreak
Indukcyjnie 
\bigbreak
Sprawdzam założenie indukcyjne dla n=2 twierdzenie jest prawdziwe, bo 

\(n=2^1\)
\bigbreak
Dla dowolnego m należącego do zbioru liczb naturalnych i m>2 niech twierdzenie będzie prawdziwe dla wszystkich n, \( 1<n<m \)
\bigbreak
Jeśli m jest liczbą pierwszą to twierdzenie zachodzi, ponieważ liczba pierwsza ma jedynie dzielnik, który jest liczbą pierwszą.
\bigbreak
Gdy liczba m jest złożona, to m można zapisać w postaci \(m=m_{1}*m_{2}\), więc gdy \(1<m_{1},m_{2}<m\) to na mocy założenia indukcyjnego każde z \(m_{1}\) i \(m_{2}\) jest skończonym iloczynem liczb pierwszych. Stąd wynika, że m też jest takim iloczynem.

\end{document}